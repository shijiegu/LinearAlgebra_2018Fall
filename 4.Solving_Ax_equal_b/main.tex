\documentclass[11pt]{article}

\usepackage[margin=1in]{geometry}
\usepackage{fancyhdr}
\pagestyle{fancy}
\usepackage{amsmath, amsfonts, bm}
\usepackage{amssymb}
\usepackage{graphicx}
\usepackage{indentfirst}
\newcommand\m[1]{\begin{bmatrix}#1\end{bmatrix}} 

\date{\vspace{-5ex}}
\title{\vspace{-5ex} Solving $A\bm{x}=\bm{b}$ \vspace{-5ex}}
\lhead{Linear Algebra Section B}
\rhead{Gu, Oct 25, 2018}

\begin{document}
{\let\newpage\relax\maketitle}
\maketitle
\thispagestyle{fancy}

\begin{enumerate}
\item $\bm{b}=\mathbf{0}$
\begin{enumerate}
\item (Matrix notion in displaying the answer)\\
Find the solution, when $A=\m{1 & 3 & 10\\2 & 6 & 20\\3 & 9 & 30}$\\
For this example, stop once you have finished elimination.\\
Now, what does it mean to have this form? You should know that what is behind this matrix form is essentially an equation. What is the solution?\\
\\
We can also express the same set of points using a matrix notation.\\
Why did we set 1's and 0's? It directly gives you independent answers.\\
\item (Some tricks in solving)\\

\fbox{If $R=\m{\bm{I} & \bm{F}\\ \bm{0} & \bm{0}}$, then $N=\m{\bm{-F} \\ \bm{I}}$ is the solution for $RN=0$.}\\
Find the solution, when $A=\m{1 & 0 & 3 & 2 & -1\\0 & 1 & 0 & 4 & -3\\1 & 1 & 3 & 6 & -4}$\\
\\
\\
What if $A=\m{1 & 3 & 0 & 2 & -1\\0 & 0 & 1 & 4 & -3\\1 & 3 & 1 & 6 & -4}$, that is, I switched the second and the third column?\\

\end{enumerate}

\item $\bm{b} \neq 0$
\begin{enumerate}
\item Solve $A\bm{x}=\bm{b}$ when $A=\m{1 & 3 & 0 & 2\\0 & 0 & 1 & 4\\1 & 3 & 1 & 6},\bm{b}=\m{-1\\-3\\-4}$\\
The solution is made up of any vectors that can give $\bm{b}$ on the right side plus any vectors that produces zeros on the right side.\\
(0) For a solution to exist, zero rows in $R$ must also be zero in $d$, which is essentially saying that $\bm{b}$ needs to be in the column space of $A$.\\
(1) Only look at rank columns, find the solution. \\
(2) Now, add in null space (sometimes null space contains only the zero vector, so you can skip this step.)\\
\\
\\
Note: the particular solution is not multiplied by any arbitrary constant.

\item Summary of possibilities for linear equations for a matrix with $m$ rows and $n$ columns.
(1)$R=\m{I} (r=m=n)$, 1 solution\\
(2)$R=\m{I & F} (r=m<n), \infty$ solutions\\
(3)$R=\m{I \\ \bf{0}} (r=n<m)$, 1 solution or 0 solution\\
(4)$R=\m{I & F\\ \bf{0} & \bf{0}} (r<m,r<n), \infty$ solutions or 0 solution\\

\textbf{Exercise}(and a useful fact in Section3.4)\\
True or false: Any set of $n$ vectors in $\mathbf{R}^m$ must be linearly dependent if $n>m$.

\textbf{Exercise}(Strang 3.3-22)\\
If $A\bm{x}=\bm{b}$ has infinitely many solutions, why is it impossible for $A\bm{x}=\bm{B}$ (new right side) to have only one solution? Could it have no solution?\\
\\
\\
\textbf{Exercise}(Strang 3.3-B)\\
In each of the description of the solution to $A\bm{x}=\bm{b}$ below, infer whenever you can about $\bm{A}$'s critical numbers, $\bm{m}$, $\bm{n}$, $\bm{r}$ and possibly about $\bm{b}$.\\
(a) There is only one solution.\\
(b) $\bm{x}=\m{2\\1}+c\m{1\\1}$\\
(c) $\bm{x}=\m{1\\1\\0}+c\m{1\\0\\1}$\\
(d) No solution\\
(e) There are infinitely many solutions.\\

\end{enumerate}




\end{enumerate}

\textbf{More Exercises}:
\begin{enumerate}
\item (Similar to Strang 3.3-32) Find the $LU$ decomposition of $A$ and the complete solution to $A\bm{x}=\bm{b}$:\\
$A=\m{1 & 1 & 0 & 2\\1 & 2 & 0 & 3\\2 & 3 & 0 & 5\\}=\m{1 & 1 & 0\\1 & 2 & 0\\2 & 3 & 1\\}\m{1 & 0 & 0 & 1\\0 & 1 & 0 & 1\\0 & 0 & 0 & 0\\}, \bm{b}=\m{2\\3\\5}$
\\
\\
\item (Strang 3.3-34)\\ Suppose you know that the 3 by 4 matrix has the vector $\bm{s}=(2,3,1,0)$ as the only special solution to $A\bm{x}=\bm{0}$, what is the row reduced echelon form $R$ of $A$?
\\
\item (Strang 3.3B)\\
Factor $A$ into $A=\bm{u}\bm{v}^{T}$, if $R$ is available.\\ $A=\m{1 & 1 & 0 & 2\\1 & 2 & 0 & 3\\2 & 3 & 0 & 5\\}=\m{1 & 1 & 0\\1 & 2 & 0\\2 & 3 & 1\\}\m{1 & 0 & 0 & 1\\0 & 1 & 0 & 1\\0 & 0 & 0 & 0\\}=E^{-1}R$\\
\\
\\
\item (Strang 3.2-48) \\
Suppose column $j$ of $B$ is a combination of previous columns of $B$. Show that column $j$ of $AB$ is the same combination of previous columns of $AB$. Then $AB$ cannot have new pivots columns, so rank($AB$)$\leq$ rank($B$). Does rank($AB$)$\leq$ rank($A$)?

\end{enumerate}
\end{document}