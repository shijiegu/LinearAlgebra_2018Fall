\documentclass[11pt]{article}

\usepackage[margin=1in]{geometry}
\usepackage{fancyhdr}
\pagestyle{fancy}
\usepackage{amsmath}
\usepackage{amssymb}
\usepackage{graphicx}
\usepackage{indentfirst}
\newcommand\m[1]{\begin{bmatrix}#1\end{bmatrix}} 

\date{\vspace{-5ex}}
\title{\vspace{-5ex} Matrix Multiplication, Inverse, Elimination \vspace{-5ex}}
\lhead{Linear Algebra Section B}
\rhead{Gu, Oct 11, 2018}

\begin{document}
{\let\newpage\relax\maketitle}
\maketitle
\thispagestyle{fancy}

\begin{enumerate}
\item Matrix multiplication in "dot product"\\
The $i$th component of $Ax$ is $a_{i1}x_1 + a_{i2}x_2 + ... + a_{in}x_n$ \\
In more compact notation: $a_i=\sum_{j=1}^{n} a_{ij}x_j$

\item Matrix multiplication in "column row"\\
$AB$ = Columns $1,..., n$ of $A$ multiply rows $1,..., n$ of $B$, and add up.\\
Example 1:
$\begin{bmatrix} 0 \\ 1 \\ 2\\ \end{bmatrix} \begin{bmatrix} 1 & 2 & 3 \end{bmatrix} = \begin{bmatrix}
    0  & 0 & 0 \\
    1 & 2 & 3 \\
    2  & 4 & 6\\
\end{bmatrix}$\\
Example 2:
$\m{ 1 & 4 \\ 1 & 5} \m{ 3  & 2 \\ 1 & 0} = \m{ 1 \\ 1}\m{3 & 2} + \m{ 4 \\ 5}\m{1 & 0} $

\item Matrix Operation Laws: \\
(1) $AB \neq BA$ in general, with some exceptions. One such is that $cI$ commute with all other matrices. Can you think of another exception?\\
\textbf{Exercise}: (Strang 2.4-6)\\
Show that $(A+B)^2$ is different from $A^2+2AB+B^2$, when $A=\m{1 & 2\\0 & 0}$ and $B=\m{1 & 0\\3 & 0}$.
Write down the correct rule for $(A+B)(A+B)=A^2+ \underline{\hspace{1cm}} + B^2$.\\
\\
\\
\textbf{Exercise}: (Strang 2.3-31)\\
Find $E_{21},E_{32},E_{43}$ to change $K$ into $U$:
$E_{43}E_{32}E_{21}\m{2 & -1 & 0 & 0\\-1 & 2 & -1 & 0\\0 & -1 & 2 & -1\\0 & 0 & -1 & 2}$\\
\\
\\
\\
(2) $(AB)C=A(BC)$\\
\textbf{Exercise} (Strang 2.4-37):
\begin{figure}[h!]
    \centering
    \includegraphics[scale=0.7]{strang37.png}
\end{figure}\\
(3) $A^3=AAA$. Note that it is A times A then times A. Not taking the power of each individual element of A.\\
(4) Block multiplication\\
To multiply A and B, make sure the cuts between columns of A match cuts in the rows of B.
Example: block elimination\\
$ E= \m{ I & 0 \\ -CA^{-1} & I} $\\
$ \m{ I & 0 \\ -CA^{-1} & I} \m{ A & B \\ C & D } = \m{ A & B \\ 0 & D-CA^{-1}B} $\\
Here is a concrete numerical example:
$\m{ 1 & 0 & 0 \\ -3 & 1 & 0 \\ -4 & 0 & 1 } \m{ 1 & x & x \\ 3 & x & x \\ 4 & x & x } = \m{ 1 & x & x \\ 0 & x & x \\ 0 & x & x } $

\item Inverse:\\
(1) some tricks on inverses:
\begin{enumerate} 
\item{2 by 2 matrices}
$\m{ a & b\\c & d}^{-1} = \frac{1}{ad-bc} \m{d & -b\\-c & a}$\\
\item{diagonal matrices}
If $A=\m{d_1 & & \\ & \ddots & \\ & & d_n}$, then $A^{-1}=\m{1/d_1 & & \\ & \ddots & \\ & & 1/d_n}$\\
\item{use meaning}\\
$E=\m{1 & 0 & 0\\-5 & 1 & 0\\0 & 0 & 1}$
$E^{-1}=\m{1 & 0 & 0\\5 & 1 & 0\\0 & 0 & 1}$\\
Caveat: extra if you use meaning:\\
\textbf{Exercise}\\
Find the inverse of $A=\m{1 & 0 & 0\\ -1 & 1 & 0 \\ 0 & -1 & 1}$
\end{enumerate} 
(2)$(AB)^{-1}=B^{-1}A^{-1}$: Think about it in terms of matrices as actors.\\
\textbf{Exercise} (Example 2-3 for "Inverse of AB" in Section 2.5 of Strang)\\
$F=\m{1 & 0 & 0\\0 & 1 & 0\\0 & -4 & 1}$\\
Calculate $F^{-1}$, $FE$, $E^{-1}F^{-1}$, and $F^{-1}E^{-1}$

\item Elimination: \\
\textbf{Exercise} (Strang 2.2-2.2B): Suppose $A$ is already a  triangular matrix  (upper triangular or lower triangular). Where do you see its pivots? 

\end{enumerate}

\textbf{More Exercises}:\\
1.(Strang 2.3-12)
Multiply these matrices:\\
(1) $\m{0 & 0 & 1\\ 0 & 1 & 0 \\ 1 & 0 & 0} \m{1 & 2 & 3\\ 4 & 5 & 6 \\ 7 & 8 & 9} \m{0 & 0 & 1\\ 0 & 1 & 0 \\ 1 & 0 & 0}$ \hspace{3cm} (2)$\m{1 & 0 & 0\\-1 & 1 & 0\\-1 & 0 & 1} \m{1 & 2 & 3\\1 & 3 & 1\\1 & 4 & 0}$\\
\\
\\
\\
\\
2.(Strang 2.2-30) If the last corner entry is $A(5,5) = 11$ and the last pivot of $A$ is $U(5,5) = 4$, what different entry $A(5,5)$ would have made $A$ singular?
\\





\end{document}